%SAVEDAT= 1453457093
\documentclass[11pt, a4paper]{article} 
\usepackage{slashed,jheppub,multirow,relsize,soul}
\usepackage[normalem]{ulem}
\usepackage{color}
\usepackage{subcaption}
\usepackage{mathabx}
\usepackage[utf8]{inputenc} 

\newcommand{\refeq}[1]{Eq.~(\ref{#1})}
\newcommand{\refeqs}[2]{Eqs.~(\ref{#1})~and~(\ref{#2})}
\newcommand{\refeqss}[3]{Eqs.~(\ref{#1}), (\ref{#2})~and~(\ref{#3})}
\newcommand{\reffig}[1]{Fig.~\ref{#1}}
\newcommand{\reffigs}[2]{Figs.~\ref{#1}~and~\ref{#2}}
\newcommand{\refsec}[1]{Section~\ref{#1}}
\newcommand{\refapp}[1]{Appendix~\ref{#1}}
\newcommand{\reftab}[1]{Table~\ref{#1}}
\newcommand{\refref}[1]{Ref.~\cite{#1}}
\newcommand{\refrefs}[2]{Refs.~\cite{#1}~and~\cite{#2}}

\def\abelian{abelian}
\def\nonabelian{non-abelian}
\def\lagrangian{lagrangian}
\def\eg{\emph{e.g.}}
\def\ie{\emph{i.e.}}
\def\aka{\emph{a.k.a.}}
\def\muboone{MicroBooNE}
\def\icarus{Icarus}
\def\minerva{MINER$\nu$A}
\def\ster{\ensuremath N}

%%%%%%% A few editorial macros. %%%%%%%

\newcommand{\lorem}{ \textcolor[rgb]{0.8,0.8,0.8}{Lorem ipsum dolor sit amet, constetur
adipiscing elit, sed do eiusmod tempor incididunt ut labore et dolore magna
aliqua. Ut enim ad minim veniam, quis nostrud exercitation ullamco laboris nisi
ut aliquip ex ea commodo consequat. Duis aute irure dolor in reprehenderit in
voluptate velit esse cillum dolore eu fugiat nulla pariatur. Excepteur sint
occaecat cupidatat non proident, sunt in culpa qui officia deserunt mollit anim
id est laborum.}}

\newcounter{CommentCount}
\setcounter{CommentCount}{1}

\newcommand{\marcom}[2]{\textsuperscript{\textcolor{#1}{\theCommentCount}}\marginpar{\textsuperscript{\textcolor{#1}{\theCommentCount}}\textcolor{#1}{{\small#1: #2}}}\stepcounter{CommentCount}}

\newcommand{\newtext}[2]{\textcolor{#1}{\ul{#2}}}

% Add your own colour down here... 
\definecolor{PB}{rgb}{0.9,0,0}
\definecolor{MARK}{rgb}{0.612, 0.153, 0.69}

%%%%%%%%%%%%%%%%%%%%%%%%%%%%%%%%%%%%%%%


\title{Decays of MeV sterile neutrinos at the Fermilab Short-Baseline Neutrino complex}

\author{Peter Ballett,}
\author{Silvia Pascoli}
\author{and Mark Ross-Lonergan}

\affiliation{Institute for Particle Physics Phenomenology, Department of
Physics, Durham University, South Road, Durham DH1 3LE, United Kingdom}

\emailAdd{peter.ballett@durham.ac.uk}
\emailAdd{silvia.pascoli@durham.ac.uk}
\emailAdd{mark.ross-lonergan@durham.ac.uk}

\abstract{
%
We study the sensitivity of the Short-Baseline Neutrino (SBN) programme at
Fermilab to $\mathcal{O}$(MeV) sterile neutrino decay-in-flight in both a minimal extension
and more generic beyond the Standard Model scenarios. We provide estimates for
the bounds that the the full SBN facility can be expected to place on the
parameter spaces of these models. We show that the facility can in many cases
extend existing bounds on $\ster \rightarrow \nu_\alpha e^+ e^-$ and $\ster
\rightarrow l^- \pi^+$ whilst, due to the strong particle identification
capabilities of liquid-Argon technology, also place bounds on often neglected
channels $\ster \rightarrow \gamma \nu_\alpha$ and $\ster \rightarrow
\nu_\alpha \pi^0$, which may be enhanced in a non-minimal model. We show that the if the light detection system employed by SBN has a sufficiently good timing resolution, it in principle allows for observable structure in the timing distributions of events relative to the beam-buckets and we
assess the regions of parameter space for which such a measurement could be expected to be able to place model independent bounds on the sterile mass.
%
%We then compare the signals of decays in flight to other models with sterile
%neutrinos and show that significant differences exist between the spectra of a
%decay in flight signal and alternative models motivated by the MiniBooNE
%excess.
%
}

\begin{document} 

\maketitle

\section{Introduction}

The neutrino sector of the Standard Model (SM) is known to be incomplete. The
observation of oscillatory behaviour between neutrino flavour states suggests a
set of mass matrices with off-diagonal terms in the flavour basis. There are
many ways that have been suggested in the literature to explain the lightness of neutrino masses
from a theoretical perspective, ranging from the ever popular see-saw scenarios
\cite{Minkowski:1977sc, GellMann:1980vs, Mohapatra:1979ia} to radiative mass
generation \cite{Zee:1980ai,Babu:1988ki} or even more involved constructions
such as neutrino masses from extra-dimensions \cite{ArkaniHamed:1998vp}. However, it will ultimately be the role of phenomenology to find ways to
distinguish between potential candidate models, and see what can be deduced about the
completion of the neutrino sector from the continued analysis of the neutrino sector in contemporary experiments.
%
Not all models of neutrino masses, however, lend themselves to such direct experimental
searches. For example, the canonical Type-I see-saw \cite{Minkowski:1977sc,
GellMann:1980vs, Mohapatra:1979ia} suggests the existence of new particles with
masses  $\approx 10^{12}$--$10^{15}$ GeV, well beyond the limits of what can be
directly produced using modern experimental techniques. An oft recurring feature among extensions of the SM which account for neutrino
masses is the presence of sterile neutrinos, SM-gauge singlet fermions which
couple to the active neutrinos via Yukawa interactions. After electoweak
symmetry breaking, these particles are coupled bilinearly to the active
neutrino fields, and upon diagonalisation, we find an extended neutrino sector
including new states with mixing suppressed gauge interactions. \emph{A priori} the
mass and interaction scales for these particles can span many orders of
magnitude, and they can be associated with wide range of distinct observable
phenomena. 
%
One of the most well-known examples of such an effect is the short-baseline
oscillation signatures associated with a sterile neutrino with a mass around
the eV scale (see \eg\ \refref{Gariazzo:2015rra} for a recent review). This
state would produce oscillatory effects over short $\mathcal{O}(100)$m distances and has been
invoked to solve the anomalies found at some short-baseline oscillation
experiments \cite{Aguilar:2001ty,Aguilar-Arevalo:2013pmq,AguilarArevalo:2008rc}.  Explaining all anomalies in an economical fashion appears
challenging in these models \cite{Kopp:2013vaa,Conrad:2012qt}, but more data
would be needed before a decision can be made as to their role in the neutrino
sector. The Fermilab SBN \cite{Antonello:2015lea} programme was primarily designed with an exhaustive search of such $\mathcal{O}$(eV) scale steriles in mind.

In this article, we study a distinctly different but complementary paradigm to the oscillatory sterile
neutrino models where sterile neutrinos exist, are light enough to be produced
in neutrino beams via meson decay, but have masses sufficiently large to
prevent oscillatory effects with the active neutrinos through loss of coherence
(see \eg\ \refref{Akhmedov:2009rb}). Due to the presence of mixing effects,
these particles are not expected to be stable and their subsequent decay products could
be observed in neutrino detectors.
%
%have been enumerated \cite{Atre:2009rg} and searched for in many previous
%experiments.
%
The focus of this paper will be on estimating the potential sensitivity of the SBN
programme for the search for sterile neutrinos of this type, with masses between $1-493$ MeV.
%
The SBN facility comprises of three separate detectors placed in the
Booster Neutrino Beam (BNB) at different (short) baseline distances: SBND
(previously known as LAr1-ND) at 110~m from the target, \muboone\ at 470~m and
ICARUS at 600~m.  All three detectors employ Liquid Argon Time Projection Chamber (LArTPC) technology \cite{Rubbia:1977} with
strong event reconstruction capabilities allowing for significantly better understood backgrounds than predecessor technologies. 
%
%We will focus in particular on \muboone\ as it the experiment has already
%began to collect data and will be first of the SBN detectors to be able to
%perform such a search.
%
Given the existing anomalies in short baseline neutrino experiments, including
MiniBooNE which also operated in the BNB, the possibility of seeing an
anomalous signature at the SBN complex should not be ignored. It is therefore
timely to ask how we can extract the physical content of such an excess, and we
present ways that such decaying steriles can be identified or excluded.

The reconstruction \cite{Church:2013hea,Marshall:2015rfa}, energy resolution\cite{Sorel:2014rka} and excellent calormetric particle identification capabilities of LAr \cite{Antonello:2012hu} technology means the SBN programme provides an ideal scenario to study the
decay in flight of sterile neutrinos. It allows for the
investigation of channels which are often unbounded in similar experiments, due
to larger backgrounds or indistinguishable signals, such as the ability of distinguishing electrons and photon EM showers by their rate of energy loss, $dE/dx$, in the first 3cm of ionizing track \cite{szelc:2007}. Furthermore, as we discuss in \ref{sec:timing}, if a sufficiently good timing resolution of scinitillation light is achieved, the timing structure of sub-luminal steriles can be utilised as both a beam related background rejection mechanism and to further aid in mass measurement.

\begin{table}[t!]
\centering
\begin{tabular}{| l || l | l | l | l |}
	\hline
	& PS-191 & SBND & $\mu$BooNE & ICARUS \\ \hline \hline
	POT	& $0.86 \times 10^{19}$	& $6.6 \times 10^{20}$	&	$13.2 \times 10^{20}$     &  $6.6 \times 10^{20}$ \\ \hline
	Volume	& $216\text{m}^3$	&	$80\text{m}^3$	&	$62\text{m}^3$	     &   $340\text{m}^3$	\\ \hline
	$\text{Baseline}^{-2}$	& $(128 	\text{m} )^{-2}$	&$(110 \text{m} )^{-2}$	&	$(470 \text{m} )^{-2}$			     & $(600 \text{m} )^{-2}$	  \\ \hline
Ratio/PS-191 & - 	& 38.5 	& 3.3	& 5.5\\ \hline
	S/$\sqrt{B}$ Ratio & - 	& 16.3 	& 1.8	& 1.1\\ \hline
\end{tabular}

\caption{\label{tab:exposure} A comparison of the relative exposure at each SBN detector
compared to PS-191. One would expect all  three SBN detectors to see increased
numbers of events than PS-191 did, with SBND seeing the largest enhancement of a
factor of $38.5$. The final row takes into account the scaling in masses
leading to increased backgrounds, although the achievable reconstruction of LAr
should reduce these significantly.}

\end{table}

In the mass range of interest for SBN, the strongest
bounds on steriles which mix with electron and muon neutrinos come from PS-191
\cite{Bernardi:1985ny}, a beam dump experiment which ran at CERN from 1983 to
1985. 
%
PS-191 was constructed from a helium filled flash chamber decay region, followed by interleaved iron plates and EM calorimeters. It was located 128~m downstream of a Beryllium target and $2.3^\circ$ (40 mrads)
off axis, obtained $0.86 \times 10^{19}$ POT over the course of its run-time,
and had a total detector volume of $6\times3\times12 = 216 \text{m}^3$. We can
estimate the sensitivity of the three SBN detectors and how they will compare
to PS-191 by estimating the experiments' \emph{exposure}, defined here as POT
$\times$ Vol $\times$ $R^{-2}$. We compare the three facilities to PS-191 in
\reftab{tab:exposure}, which shows that all detectors of the SBN complex expect
a larger exposure, with SBND seeing the largest enhancement by a factor of
around $40$. 
%
In addition to the larger exposure, there is also an enhancement of the
expected decay events at SBN due to its lower beam energy. The steriles at SBN
are produced by the 8 GeV BNB beam and have a softer spectrum than those
produced by the 19.2 GeV proton synchrotron beam used at PS-191. As we 
discuss in more detail in \refsec{sec:decays}, the probability of sterile
neutrino decay scales as $1/E_\nu$, and we would therefore expect the SBN
facilities to see more events than PS-191 for equivalent exposures.

PS-191 was purposefully built to search for decays in flight of
heavy fermions, and so as to minimize the background induced by active neutrino
scattering, the total mass of the detector (and therefore number of target
nuclei) was chosen to be small (approximately $20$ ton). Conversely, the SBN detectors were designed to search for neutrino interactions and thus have significantly larger
masses ($112$, $66.6$ and $476$ tons respectively).  We therefore expect SBN to
not only see a greater number of decay events than PS-191, but to also see a
dis-proportionally greater background for a given exposure.
%
Therefore, to understand the sensitivity of SBN to heavy sterile decay, the
modest background rejection capability must be sufficiently motivated.
Ultimately, the loss of sensitivity will depend on the skill of the analysis
and the attainable degree of background suppression. We will discuss this in
more detail in \refapp{app:bg}.

It is clear from \reftab{tab:exposure}, that SBND has the best comparison to
PS-191. Indeed, we will find that the majority of the sensitivity of the full
SBN data set will come from SBND thanks to its higher event rates. However, due
to the staged approach of SBN, data from \muboone\ (which is already collecting
data) is expected well before that of SBND or ICARUS (expected to begin
operations in 2018). For this reason, we will consider all detector sites in
our analysis, and comment on the comparison between signals at \muboone\ and at
the full SBN facility.

Our discussion starts in \refsec{sec:decays} with an overview of sterile
neutrino decay in minimal and non-minimal models. We then present the details
of our simulation in \refsec{sec:simulation} and show illustrative event
spectra for some channels of interest. In \refsec{sec:sensitivities}, we show the
exclusion contours that SBN could place on the model in the absence of a
signal. We then study how the event timing information could be used to
constrain the particle masses if a positive signal were indeed detected. We make some
concluding remarks in \refsec{sec:conclusions}.

\section{\label{sec:decays}Sterile neutrino decay}

The most general renormalizable \lagrangian\ extending the SM to include a
single new gauge-singlet fermion $\ster$ is given by
%
\begin{align}   \mathcal{L} = \mathcal{L}_\text{SM} +
\overline{\ster}i\slashed{\partial}\ster+ \frac{\mu}{2}
\overline{\ster^\text{c}}\ster  + y_\alpha\overline{L}_\alpha\tilde{H} \ster +
y_\alpha^* \overline{\ster}\widetilde{H}^*L_\alpha\label{eq:minimallag}
\end{align}
%
where $y_\alpha$ denote Yukawa couplings and $\mu$ a Majorana mass term for
$\ster$. The extension to multiple new fermions involves promoting $y$ and
$\mu$ to matrices with indices for the new states, but will offer no real
phenomenological difference in the current work.\footnote{The minimal single
$\ster$ extension does not allow for the observed masses of the neutrinos, as
the mass matrix is rank 1. We assume that an appropriate extension has been
introduced to satisfy neutrino oscillation data while introducing no new
dynamics at the lower energy scales of interest.} Much work has been done
understanding the phenomenology of such novel neutral states, which varies
significantly over their large parameter spaces. 
%
Lagrangians similar to this have been used in the literature for a wide range
of purposes. If the new particle has a mass around $10^{15}$ GeV it could
provide a natural way to suppress the size of active neutrino masses through
the Type I or III see-saw mechanisms \cite{Minkowski:1977sc, GellMann:1980vs,
Mohapatra:1979ia}. A lighter neutral fermion, with a mass around the keV scale,
remains a promising dark matter candidate \cite{Adhikari:2016bei}. A synthesis
of these ideas is found in the so-called $\nu$MSM which simultaneously can
explain dark matter, neutrino masses and successful baryogenesis
\cite{Asaka:2005pn}. 
%
If the sterile gets lighter still, with masses at the eV scale or below,
important observable effects might exist in neutrino oscillation experiments.
Indeed, such particles have been proposed to alleviate short-baseline
oscillation anomalies; although, no minimal solution seems to provide a
compelling universal improvement to the current data \cite{Kopp:2013vaa,Conrad:2012qt}.

For steriles which are light enough to be produced in a neutrino beam,
%
%typically with masses below the $D$-meson mass scale $m_{\ster} \lesssim
%2$~GeV, 
%
there is a qualitative divide in the phenomenology somewhere between keV and eV
energies\footnote{The precise mass range depends on details of the process
under consideration.}. If the sterile neutrinos are massive enough for their
mass-splittings between the light neutrinos to be larger than the wavepacket
uncertainty associated with the production mechanism, they no longer oscillate
\cite{Akhmedov:2009rb}.  
%
Once oscillation is suppressed, neutral particles produced in the beam will
propagate towards the detector and may be observed by their subsequent decay
into SM particles. Experiments seeking to measure such decays are generally
known as beam dump experiments where proton collisions with a target produce
particles to be observed down-wind of the source \cite{CooperSarkar:1985nh,
Bergsma:1985is, Vaitaitis:1999wq, Bernardi:1985ny, Bernardi:1987ek,
Anelli:2015pba, Alekhin:2015byh}. It has been pointed out that the difference
between a beam dump and a conventional neutrino beam is more a matter of
philosophy, and we can expect many experiments to have some sensitivity to
novel heavy states \cite{Gorbunov:2007ak, Asaka:2012bb, Adams:2013qkq}. 
%
For the BNB, we can estimate the mass at which this decoupling occurs as
follows: the decay pipe for BNB is around $50$~m in length, which is
considerably shorter than the decay lengths of the mesons in the beam.
Therefore we assume that this length defines the wavepacket width at
production.  The relevant parameter is the decoherence parameter
\cite{Akhmedov:2009rb, Hernandez:2011rs}
%
$\xi = 2\pi \frac{\lambda_\text{d}}{\lambda_\nu},$
%
where $\lambda_\text{d} = 50$~m and $\lambda_\nu$ is the standard neutrino
oscillation length $\lambda_\nu = \Delta m^2/4E_\nu$. For $\xi\gg1$ the wave
packet is insufficiently broad to accommodate a coherent superposition of the
heavy and light neutrino states. We estimate that this occurs for the BNB at 
%
$ \Delta m^2 \gtrsim 0.01~\text{keV}^2.$
%
In this article we study the region of MeV scale sterile masses. These are
heavy enough to forbid oscillatory effects while light enough to be have
significant production from the meson decays associated with a conventional
neutrino beam. 

\begin{figure}[t]
\centering
\begin{subfigure}{.5\textwidth}
  \centering
  \includegraphics[width=\linewidth]{figures/BR_notlog_square.pdf}
\end{subfigure}%
\begin{subfigure}{.5\textwidth}
 \centering
\includegraphics[width=\linewidth]{figures/BR_notlog_square2.pdf}
\end{subfigure}
\caption{\label{fig:branchingratios}The branching ratios for sterile neutrino
decays in the minimal 3 sterile SM extension, with masses between 1 MeV and 0.5
GeV. (Left) Assumes equal mixing with all active flavours whereas (Right) assumes either $U_{e4}$ only (solid lines) or $U_{\mu 4}$ only (dashed lines).
Heavier steriles will not be produced in great numbers in the BNB beam due
to the very small componant of heavier mesons in the secondary beam \cite{AguilarArevalo:2008yp}.Note that below 100 MeV, $\ster
\rightarrow \nu_\alpha e^+ e^-$ is by far the most dominant visible decay.}
\end{figure}



The observable signature in this model is the direct decay-in-flight of the new
fermion into SM particles. In the minimal \lagrangian\ in
\refeq{eq:minimallag}, the only direct couplings to new sterile fermions are
neutrino--Higgs interactions. However, these couplings generate off-diagonal
neutrino bilinears below the electroweak symmetry breaking scale, which leads
to mass mixing between the $4$ (or more) flavours of neutrinos. This generates
production and decay mechanisms of many kinds for the state $\ster$ through
mass insertion on an active neutrino fermion line in a gauge mediated process.
These decays have been studied extensively in the literature \cite{Atre:2009rg}
and depend only on the size of neutrino mixing to various flavours,
parameterized by the elements of an extended $4\times4$ PMNS matrix,
%
$U_{\alpha 4}$ for $\alpha \in \{e,\mu,\tau\}$, 
%
and the mass of the state $\ster$ itself. The branching ratios for these decays
are shown in \reffig{fig:branchingratios} as a function of mass for the case if
the new state mixes with all flavours of active neutrino equally $U_{e4}=U_{\mu
4}=U_{\tau 4}$. Less democratic flavour patterns are of course possible and, in
the extreme case, might lead to certain channels being forbidden or vastly
suppressed. For example, without the coupling to the electron $U_{e4}=0$, the
dominant channel for masses $m_{\pi^0}~\lesssim m_\ster \lesssim
m_{\mu}+m_{\pi^\pm}$ will be into a neutral pion and a neutrino.
%

This suppression of couplings in turn affects the possible production
mechanisms for the $\ster$ particles. In a conventional neutrino beam, most
neutrinos are derived from meson decay (or secondary $\mu^\pm$ decays). If
$U_{e4}=U_{\mu 4}=0$, these decays with a mass insertion for the sterile
neutrino are impossible for pions or kaons. For this reason, we will mainly
focus on mixing with the first two generations. This parameter space will be
probed by working at higher energies, where the neutral fermions can be
produced by decays of charmed mesons such as $D^\pm$, by the SHiPS experiment
\cite{Alekhin:2015byh, Anelli:2015pba}.   
%
We focus on five decays in our study, which have the largest branching ratios
of all channels with visible decay products over the mass range producable from
pion and kaon decay, $m_{\ster} \lesssim m_K$. These are shown schematically in
\reffig{fig:decaydiagrams}.

\begin{figure}[t]
\begin{align*}
\pi^\pm / K^\pm \longrightarrow \overbrace{l^\pm \ster}^{\text{via } U_{e4} \text{ or } U_{\mu 4}}  &\longrightarrow \left. l^\pm \pi^\mp, \qquad \right\} \text{via } U_{e4} \text{ or } U_{\mu 4}\\
														    & \drsh \nu_\alpha e^+ e^-,\\[\jot]
					   & \drsh \nu_\alpha \gamma, \hspace*{3em}
		\smash{\left.\begin{array}{@{}c@{}}\\[\jot]\\[\jot]\\[\jot]\end{array}\right\}}\text{via } U_{e4}, U_{\mu 4} \text{  or  } U_{\tau 4}\\[\jot]
						& \drsh \nu_\alpha \pi^0.
\end{align*}

\caption{\label{fig:decaydiagrams}The production and decay channels considered in this study.}

\end{figure}

\subsection{Minimal model}

We define the minimal sterile neutrino model by the Lagrangian in
\refeq{eq:minimallag}. This corresponds to the best known model of sterile
neutrino phenomenology: the UV-complete Type I see-saw (and its low-scale
variants). Decays of sterile neutrinos in such a model have been studied in
\refref{Atre:2009rg}, and we briefly summarize the results most important for
the present study.

For sterile neutrino masses less than the pion mass, the dominant visible decay
will be into an electron-positron pair as can be seen from
\reffig{fig:branchingratios}. This is true regardless of the flavour structure
of the mixing matrix; although, the decay rate is not flavour-blind, due to the
presence of additional CC diagrams relevant only for the eletron-sterile
mixing. The decay rate for this channel is given by 
%
\begin{align*} \Gamma\left(\ster\to \nu_\alpha e^+e^-\right) =
\frac{G_\text{F}^2m_\ster^5}{96\pi^3}\left|U_{\alpha 4}\right|^2&\left[\left( g_Lg_R + \delta_{\alpha e}g_R\right)I_1\left(0,\frac{m_e}{m_\ster}, \frac{
m_e}{m_\ster}\right)\right.\\ 
&\left.\qquad + \left(g_L^2 + g_R^2 + \delta_{\alpha e}(1+2g_L)\right)I_2\left(0,\frac{m_e}{m_\ster},\frac{m_e}{m_\ster}\right)\right],  \end{align*}
%
where $I_1(x,y,z)$ and $I_2(x,y,z)$ are integrals over phase space such that $I_1(0,0,0) = 1$ and $I_2(0,0,0) = 0$. Further details of the decay rates used in this work are given in
\refapp{app:decayrates}.
%
Decays of this type would generally fall into two categories of events at a LAr
detector, and accordingly we divide our analysis. The first event sample will
attempt to measure events where two tracks are resolved and two
electron-induced electromagnetic showers are observed.
%
The second sample consists of events for which the identification of two
electron-like showers is impossible. In this case, the signal would be
identical to a single photon pair-conversion. We expect a larger background for
this scenario, but as we will show, we get sizable event numbers in this
channel due to the sterile neutrino's high energy, and a tight cut on the
angular distribution can make it sensitive to sterile decays.

Although the electron-positron channel dominates the decays at these masses,
another interesting channel is given by the radiative decay
$\ster\to\nu_i\gamma$ would generate an interesting single photon signal
\cite{PhysRevD.25.766}. In the minimal model the decay occurs via a
charged-lepton/$W$ loop and has a rate given by
%
\[ \Gamma(\ster\to\nu_i\gamma) = \frac{G_\text{F}^2m_\ster^5 |U_{\alpha
4}|^2}{192 \pi^3} \left( \frac{27 \alpha}{32 \pi} \right). \]
%
This decay channel is significantly suppressed by the light mass of the
sterile, the mixing elements and the loop factor. It can be estimated at
around $\Gamma(\ster\to\nu_i\gamma)/(\text{GeV}) \approx 10^{-20}
(m_\ster/\text{GeV})^5$.  

Low-mass sterile decay is dominated by the photon and electron-positron decays
up until the pion mass threshold. In the few keV after this threshold, two
further decays become possible: $N\to\nu \pi^0$ and $N\to e\pm\pi^\mp$. 
%
The decay rate for the first process is given by
%
\[ \Gamma\left(\ster \to \nu_i \pi^0\right) =
\frac{G_\text{F}^2f_\pi^2m_\ster^3\left|U\right|^2}{64\pi} \left[1-\left(
\frac{m_\pi}{m_\ster} \right)^2\right].  \]
%
where $\left|U\right|^2 = \sum_{\alpha}\left|U_{\alpha 4}\right|^2$.
% 
The decay into a charged pion and an electron is an important channel, and one
of the most constrained in direct decay experiments due to its clear visible
two-body signal.  Its decay rate is similar to the $N\to \nu
\pi^0$ channel, with the addition of a CKM matrix element from the $W$-vertex,
%
\begin{align} \Gamma\left(\ster\to l^\pm\pi^\mp\right) =
\left|U_{l4}\right|^2\frac{G_\text{F}^2f_\pi^2 |V_{ud}|^2
m_\ster^3}{16\pi}I\left(\frac{m_l^2}{m_\ster^2} ,
\frac{m_\pi^2}{m_\ster^2}\right) , \label{eq:chargedlep_decayrate}\end{align}
%
where $I(x,y)$ is a kinematic function which away from the production threshold
provides a small suppression $\sim 0.5$. Details are in \refapp{app:decayrates}.

The $N\to e^\pm\pi^\mp$ channel dominates the branching ratios in the mass
range $130 \gtrsim m_\ster  \lesssim 235$ MeV. However, as it is mediated by a
$W$ boson, in the absence of $U_{e4}$ mixing, this decay would be impossible
and the decay into a neutral $\pi^0$ and a light neutrino would be dominant.
Once the mass of the sterile fermion is above $m_\ster \gtrsim 235$ MeV,
another charged-lepton pion decay opens up, $\mu^\pm\pi^\mp$. This is another
strongly constrained channel, and its decay rate is already given in
\refeq{eq:chargedlep_decayrate} with $m_l = m_\mu$. 

We have plotted the relevant branching ratios of $N$ in
\reffig{fig:branchingratios}. In addition to those that we have discussed so
far, there are several further decays possible $\ster\rightarrow \nu_\alpha
\mu^+ \mu^-$ and $\ster \rightarrow \nu_\alpha e^+ \mu^-$. In the minimal
model, these decays are subdominant with respect to other decays, and we will
not discuss them further.

\subsection{Non-minimal models}

An important feature of the minimal model that we have discussed so far is the
all-or-nothing approach to decay rates: all decay processes of the sterile
neutrino are weak processes scaled by a mixing matrix element, and it is the
three mixing matrix elements $U_{\alpha 4}$ alone which dictate the magnitude
of all the decay rates. This is a great asset when trying to constrain the
minimal model, for example it allows us to convert any non-observation of a
signal into a bound on the mixing matrix, and allows us to reinterpret bounds
on any one decay rate onto the expected decay rates for other channels, such as
$\ster \rightarrow \nu_\alpha e^+ e^-$ non-observation being used to bound
$\ster \rightarrow \nu_\alpha \pi^0$ despite no experiment to-date reporting
the results of such a search. 
%
However, this asset becomes a bias when used in a general search for novel
neutral states. Although such low-scale see-saws are a viable region of
parameter space, they lack a theoretically appealing mechanism to explain the
sub-electroweak sterile neutrino mass scale nor the sizes of neutrino masses.
Alternative models exist which feature a light neutral particle and a mechanism
to generate tiny neutrino masses, but they rely on extended field content or
interactions.
%
Indeed it has been stressed before that the discovery of a light sterile
neutrino would necessitate not just the addition of new neutral fermions to the
SM but would be a sign of the existence of some non-trivial new physics
\cite{delAguila:2008ir}. With these ideas in mind, we briefly consider the
properties of a light neutral fermion in a low-energy effective theory.  

The effective field theory of a SM extension involving new sterile fermions has
been presented at dimension 5 \cite{delAguila:2008ir,Aparici:2009fh}, dimension
6 \cite{delAguila:2008ir} and dimension 7 \cite{Bhattacharya:2015vja}.
%
We extend the field content of the SM by a sterile fermion $N$\footnote{As
before, we focus on the single $N$ case for illustrative purposes. The
extension to a set of fields $N_i$ is uneventful.}.  


In \refref{Aparici:2009fh} the phenomenology of the effective operators at
dimension 5 are considered in detail. Along with the Weinberg operator, which
could be the source of a light neutrino Majorana mass term
\cite{Weinberg:1979sa}, the authors find two effective operators: an operator
coupling the sterile neutrino to the Higgs doublet and a tensorial coupling
between the sterile neutrino and the hypercharge field strength 
%
\[ \mathcal{L}_5 \supset \frac{c_1}{\Lambda}\overline{\ster^c}\ster(H^\dagger
H) + \frac{c_2}{\Lambda}\overline{\ster^c}\sigma^{\mu\nu}\ster B_{\mu\nu}. \] 
%
At energies below the electroweak scale, and after diagonalisation into mass
eigenstates for the neutrinos, these operators generate novel couplings, for
example a vertex allowing $\ster\to h \nu$ ($\ster_1\to h \ster_2$), $\ster\to
\nu Z$ ($\ster_1\to Z \ster_2$) and $\ster \to \nu \gamma$ ($\ster_1 \to
\ster_2 \gamma$) at a rate governed by the scale of new physics suppressing
these operators.
%
Of particular interest is the electroweak tensorial operator, which induces a
rich range of phenomena \cite{Aparici:2009fh}. In the mass range of interest in
the present work, bounds on such an operator are fairly weak: strong
constraints from anomalous cooling mechanisms in astrophysical settings
generally occur only for lower sterile neutrino masses, whilst collider bounds
only become competitive for higher masses.\marcom{PB}{We should check this.
Aparici et al. don't say what happens to LHC bounds in this region.}
%
See also \refref{Duarte:2016miz} for a discussion of decay rates in the
effective sterile neutrino extension up to dimension 6.

Although effective models of sterile neutrino decay can describe the low-scale
effects of high-scale physics, their predictions are invalid if unidentified
light degrees of freedom exist in the model. For example, models with sterile
neutrinos that also feature novel interactions can have significantly different
decay rates and branching fractions, strengthening some bounds and invalidating
others \cite{Batell:2016zod,Ballett:2016xxx}.
%
As an example, a model with a leptophilic $Z^\prime$ \cite{Foot:1994vd} could
enhance purely leptonic decay rates, such as $\ster\to \nu e^+\mu^-$, while not
enhancing semileptonic processes like $\ster\to e^\pm \pi^\mp$.
\marcom{PB}{Check this reference. It is $\mu$-$\tau$, does what I say make
sense? I follow it, seems good.}
%
Often, such models would have to be checked individually to see if they satisfy
bounds from other experiments; however, the point stands that if new physics
exists in relation to neutrino masses beyond the addition of neutral fermions,
we must allow for non-minimal interactions and decays.

With reference to our present study, the existence of non-minimal models
suggests that we should seek to bound each kinematically allowed decay mode
independently. Moreover, we should be careful about concluding that channels
are forbidden purely from the inferred bound from an experimentally constrained
channel.  
%
However, it is worth pointing out that an enlarged decay rate would not
invalidate any bound placed experimentally on a given channel. In fact,
reporting the results of the non-observation of certain decay products in terms
of a mixing matrix element in the minimal extension is mostly harmless. We can
always see this as an effective quantity. 

Secondly, in any extended model the minimal couplings are still present. This
is an unavoidable consequence of mass-mixing between the active and sterile
sectors and the interactions of the SM. Therefore, without suspicious
cancellations (or significant destructive interference) between the
mixing-derived and higher-dimensional contributions to a decay, the
non-observation of a signal in a given decay channel can still be used to bound
the mixing matrix element, and therefore the minimally-correlated bounds
\emph{do} apply to the Standard Model + mixing decay processes. 

For the reasons discussed so far, we believe it is relevant to place bounds on
the possible decays of a neutral fermion in an extended scheme which allows for
non-standard decay rates to visible particles (within the bounds of perturbativity
and other conventional model building constraints).
%
We will focus here on the posibility for enlarged neutral current decay rates
of the mostly-sterile fermion. In this scenario, the production of the sterile
state will be largely unaffected by the new physics: heavy states will be
produced in the beam from meson decay as usual. However, we allow for an
increased rate of decay for the two channels $\Gamma_{\gamma\nu}$ and
$\Gamma_{\pi^0\nu}$. 

To end this section, we point out that non-standard decay rates of sterile
neutrinos have been invoked in beam dump experiments before. In
\refref{Gninenko:2009ks,Gninenko:2010pr}, which seeks to explain the MiniBooNE
excess through the production and decay of a new sterile fermion inside the
detector. This is achieved by introducing an extremely short-lived sterile
neutrino, which is produced via a mixing-mediated neutral current interaction
inside the detector, and which decays into a single photon on a very fast
timescale. This enhancement to $\ster\to \nu \gamma$ is driven via a large
transition magnetic moment and is in addition to all the decays one has soley
due to the active-sterile mixing introduced. A large enough rate to explain
MiniBooNE requires a very large mixing with muon neutrinos, $|U_{\mu 4}|^2
\approx 10^{-3}$--$10^{-2}$  for $m_N \approx 40-80$ MeV with an enhanced
lifetime of $\tau_N \leq 10^{-9}$s.

When considering any of these enlarged decay rates, we must be careful with
existing bounds on the model, as an enlarged decay rate would affect all prior
beam dump experiments in the same way until the point where baseline dependence
becomes significant. It is instructive now to explore how to scale the bounds
presented here, or indeed any preexisting beam-dump experimental bounds, from
the minimal model to an extended model that has a enhancement in one or more
channels. To estimate this we assume the same production mechanism and rate
(i.e via standard meson decay through sterile-active mixing $\propto
|U_{\alpha}|^2$ only) along with the enhancement of a single channel by a
factor $\alpha$, such that the total decay rate is given by $\Gamma_T =
\Gamma_\text{other}+\Gamma_c (1+\alpha)$. Every beam-dump style experiment
produces two bounds on $|U_{\alpha 4}|^2$ even though many only report one, an
upper bound where decreasing decay rates produce too few decays to be
statisticall significant and lower bound corresponding to the scenario where
increasing decay rates have a vanish survival probability of reaching the
detector. To estimate these values one compares the flux folded probabilities
for an experiment of baseline(length) L($\lambda$) to decay in each scenario,
which in the limit of small $\epsilon \equiv \lambda/L \ll 1$ approximates
Lamberts equation with solutions, \begin{align*} |U_\text{extended}|^2 &\leq
\frac{-2 |U_\text{minimal}|^2}{\alpha \Gamma^\prime_c +\Gamma^\prime_T}
\mathcal{W}_{0,-1}\left[-\exp^{-\frac{\Gamma^\prime_T \kappa}{2}}
\frac{(\Gamma^\prime_c \alpha + \Gamma^\prime_T)\kappa}{2\sqrt{1+\alpha}}
\right], \end{align*} where $\mathcal{W}_{(0,-1)}$ are the two real branches of
the Lambert-W function corresponding to the two regimes of bounds in a decay in
flight experiment respectively, and primed decay rates are in the minimal model
with $\kappa \equiv L/\gamma \beta$. In the former case, $\mathcal{W}_0$ and
$\Gamma_T L \ll 1$, the bound can be reduced to the more intuitive form
\begin{align*} |U_\text{extended}|^2 &\leq
\frac{|U_{\text{minimal}}|^2}{\sqrt{1+\alpha}}.  \end{align*} These can be used
to map the upper and lower bounds presented below to any enhanced sterile
model.\\

Additonal non-terrestial measurements may also place bounds on such long lived sterile neutrinos. One such implication of additional states is they may strongly impact the success of Big-Bang Nucleosynthes(BBN) by both speeding up the expansion of the universe with their extra energy, as well as potentially modifying the spectra of active neutrinos via their subsequent decays. If, however, the sterile has suffciently short lifetime  then their effect on BBN is mitigated as the bulk of thermally produced steriles have decayed long before $T^i_\text{BBN} \approx 10$ MeV \cite{Fields:2006ga}. The strength of these bounds have been estimated conservatively for a single sterile neutrino, $m_N < m_{\pi^0}$, as \cite{Dolgov:2000jw,Dolgov:2000pj}
\begin{align*}
	\tau_\text{N} &< 1.287 \left( \frac{m_N}{\text{MeV}}\right)^{-1.828}+0.04179 \text{  s    $\qquad$  for $U_{\mu 4}$ or $U_{\tau 4}$ mixing},\\
	\tau_\text{N} &< 1699 \left( \frac{m_N}{\text{MeV}}\right)^{-2.652}+0.0544 \text{  s    $\qquad$  for $U_{e 4}$ mixing},
\end{align*}
at the 90\% C.L. Although the scenario for $m_N > m_{\pi^0}$ has not been studied in detail, an oft used bound is that $\tau_N > 0.1$s is excluded under current BBN measurements \cite{Dolgov:2000j}. In the minimal model this upper bound on the sterile lifetime can be directly mapped to a minimum bound on the active-sterile mixings $U_{\alpha 4}$'s. However, even a modest increase of decay rate, for example by additional interactions in the sterile sector leading to decays that are not mixing supressed, pushes the total sterile lifetime below 0.1s and removes the bounds completely. Similarly any bounds derived as such fold in a plethora of early universe model assumptions. As such we argue that terrestial bounds on each individual decay channel are still vital probes of the sterile sector. \\

\section{\label{sec:simulation}Short Baseline Neutrino complex}

\subsection{Simulation details}
A critical aspect of estimating the sensitivity of the SBN to sterile decays is
accurate modeling of the incoming sterile fluxes. 
%
Although novel dynamics may lead to enhanced production rates of sterile
neutrinos by unconventional means, we neglect this possibility and assume that
the sterile component of the BNB flux arises from meson (or subsequent $\nu_\mu$) decay only. This process
requires only mass-mixing from the $N$-$\nu$ Yukawa term in
\refeq{eq:minimallag}. It follows that the amplitudes for these decays are
related to those of the standard leptonic decays of mesons via an insertion of
the mixing matrix element $U_{\alpha 4}$, and to leading order in the mass of
the sterile neutrino over the meson, the $\ster$-fluxes will be a rescaling of
the fluxes for the active neutrinos. We take these active neutrino fluxes as our
input and scale them by the appropriate mixing $U_{\alpha 4}$, with an
additional kinematic factor to take into account the helicity un-suppression of
channels such as $\pi^+ \rightarrow e^+\ster$ for massive $\ster \gg m_e$. The
flux of steriles produced from the decay of a given meson M is therefore
%
\[ \phi_{\ster}(E_{\ster}) \approx \phi_{\nu_\alpha} (E_{\nu_\alpha})\vert
U_{\alpha 4}\vert^2 \frac{\rho\left( \delta_M^a , \delta_M^i
\right)}{\delta_M^a \left(1- \delta_M^a\right)^2}.  \]
%
Where $\rho(a,b)=\mathcal{F}_M(a,b) \lambda^{\frac{1}{2}}(1,a,b)$ is a
kinematic factor consisting of a term proportional to the two body phase space
factor, $\lambda(x,y,z)=x^2+y^2+z^2-2(x y+yz+x z)$ and a term proportional to
the matrix element, $\mathcal{F}_M(a,b)= a+b -\left(a-b\right)^2$, with
$\delta_M^{a(i)}=m_{l_a(\nu_i)}^2/M^2$ \cite{PhysRevD.24.1232}. This kinematic
effect for the pion and kaon can be substantial, in the case of $\pi
\rightarrow e \nu$ this factor can be as large as $10^5$, which more than compensates
for the significantly smaller intrinsic flux of $\nu_e$  inherent in the BNB, $\approx 0.52$\% \cite{AguilarArevalo:2008yp}. We only
consider steriles below the kinematic threshold of the kaon (388 MeV for
production via $|U_{\mu4}|$ mixing and 493 MeV for $|U_{e4}|$ driven
production) as although there is a small component of heavier mesons such as
the D meson which could produce heavier sterile states, they are produced in
very small numbers due to the relatively low energy protons of the BNB beam \cite{AguilarArevalo:2008yp}. As
the mass of the sterile increases, we begin to see components of the active flux
having energies less that the sterile mass, artificially loosing these events
from the beam. In order to keep the normalisation of total neutrino events
constant, before $U_{\alpha 4}$ and kinematic scaling, these events are shifted
to be above the sterile mass threshold.

The input neutrino fluxes at all three SBN detectors are calculated from published
MiniBooNE fluxes \cite{AguilarArevalo:2008yp}, after scaling by appropriate
$1/r^2$ baseline dependence, e.g $(470/540)^2 \approx 1.3$ at \muboone. This
is similarly scaled by $1/r^2$ for ICARUS at 600m, however, an additional
energy dependent flux modifier is applied for SBND at 110m to account for the
softer energy spectrum due to the proximity of the detector to the production
target \cite{Antonello:2015lea}. We consider sources of neutrinos that are
relavent including wrong sign neutrinos, smaller sub-dominant $K^+\rightarrow
\pi^+\rightarrow \nu_\alpha$ as well as other contributions, predominately from
meson decay chains initiated by meson-nucleus interactions, although all
contributions other than neutrinos from two body meson decays are scaled by
$|U_{\alpha 4}|^2$ only, not including the kinematic enhancement mentioned
above. The neutrino spectrum at \muboone\ is shown below in figure
(\ref{fig:flux_plots}). As the probability for any given sterile to subsequently decay scales as
$1/|P_{\ster}|$ the fluxes at low energy are of particular importance, in stark contrast to neutrino interaction cross sections, which scale as $E_\nu$. This leads to further kinematic discrepancies between decay-in-flight and interaction spectra. 





\begin{figure}[t]
\center
\begin{subfigure}[t]{0.5\textwidth}
\includegraphics[width=\textwidth]{figures/microBooNE_flux.pdf} 
\end{subfigure}%
~
\begin{subfigure}[t]{0.5\textwidth}
\includegraphics[width=\textwidth]{figures/microBooNE_flux_weighted.pdf}
\end{subfigure}

\caption{\label{fig:flux_plots} Left: The composition of fluxes of $\nu_\mu$
and $\overline{\nu}_\mu$ at \muboone\ with horn in positive polarity (neutrino
mode). ``Other'' refers to contributions primarily from meson decay chains
initiated by meson-nucleus interactions. Right: Fluxes weighted by the
probability to decay inside \muboone, for a sample 25 MeV sterile with equal
$|U_{e4}|^2 = |U_{\mu 4}|^2$, and the horn in neutrino-mode. Requiring that the
sterile decays has the effect of vastly increasing the importance of lower
energy bins, where traditionally cross-section induced background effects are
small.}

\end{figure}

\begin{figure}[t]
\center

\end{figure}

We simulate the event numbers and distributions at each detector site using a custom
Monte Carlo program which allows efficiencies to be taken into account due to
experimental details such as energy and timing resolution in a fully correlated
way between observables, and provides us with event level variables for use in
our cut-based analysis. 
%
Given the spectral flux of sterile neutrinos impinging on a SBN detector,
$\mathrm{d}\phi/\mathrm{d}E$, we compute the total number of accepted events in
channel ``$\text{c}$'' via the following summation,
%
\[ \ster_\text{c} = \sum_{i} \left .
\frac{\mathrm{d}\phi}{\mathrm{d}E}\right|_{E_i} P_\text{D}\left(E_i\right)
W_\text{c}\left(E_i\right),  \]
%
where $P_\text{D}(E)$ is the probability for a sterile of that energy to reach
and then decay inside the detector labelled $\text{D}$. The simplest
approximation is to ignore all geometric effects, so that every particle
travels exactly along the direction of the beam line, which gives the following
probability 
%
\[ P_\text{D}\left(E\right) = e^{-\frac{\Gamma_\text{T}L}{\gamma\beta}}\left(
1-
e^{-\frac{\Gamma_\text{T}\lambda}{\gamma\beta}}\right)\frac{\Gamma_\text{c}}{\Gamma_\text{T}},
\label{eq:prob} \]
%
where $\Gamma_\text{T}$ ($\Gamma_\text{c}$) denotes the rest-frame total decay
width (decay width into channel $\text{c}$), $m$ the mass of the sterile
neutrino, and $L$ ($\lambda$) the distance to (width of) the detector. The
combination $\gamma\beta$ is the usual special relativistic function of
velocities of the parent particle and provides the sole energy dependence of
the expression
%
\[   \frac{1}{\gamma\beta} \equiv \frac{m}{\sqrt{E^2-m^2}}. \]
%
As we are exploring a large parameter space, often this expression takes a
simplified form depending on the size of $\Gamma_\text{T}\lambda/\gamma\beta$:
%
\begin{align} 
%
\Gamma_\text{T}\lambda \ll 1\qquad&\qquad P_\text{D} \approx
e^{-\frac{\Gamma_\text{T}L}{\gamma\beta}}\frac{\Gamma_\text{c}\lambda}{\gamma\beta}
+ \mathcal{O}\left(\Gamma_\text{T}^2\lambda^2\right),\label{eq:prob_dec1}\\
%
\Gamma_\text{T}\lambda \gg 1\qquad&\qquad P_\text{D} \approx
e^{-\frac{\Gamma_\text{T}L}{\gamma\beta}}\frac{\Gamma_\text{c}}{\Gamma_\text{T}}
+ \mathcal{O}\left(\frac{1}{\Gamma_\text{T}\lambda}\right),
\label{eq:prob_dec2}
%
\end{align}
%
where the rate for slowly decaying particles can be seen to grow with detector
size until a width of $\lambda\sim\Gamma_\text{c}^{-1}$ where longer detectors
make no difference, as most steriles decay within a few decay lengths and
therefore we see a fixed fraction of the total events in our channel of
interest. 

%
Finally, the function $W_\text{c}(E)$ is a weighting factor which accounts for
all effects which reduce the number of events in the sample: for example,
analysis cuts or detector performance effects.
%
To compute these factors, we run a Monte Carlo simulation of the decays for a
large number of sample events with a given energy. Each sterile event is
associated with a decay of type $\text{c}$. We then apply experimental analysis
cuts to the decays based on our assumptions about the detector's capabilities
and backgrounds, to produce a spectrum representing the final event sample when
considering events in the bucket timing window (See \refapp{app:bg} for
details of the background analysis). The percentage of accepted events defines
the weight factor for that energy. As such the full spectral shape of the
signal is included in the rate analysis. As a consistency check of our
methodology, we also reproduce in \refapp{sec:ps191} some of the published
bounds of PS-191. 


\subsection{\label{sec:eventspectra}Event spectra}

%%%% This figure shows two spectra WITHOUT CUTS (left) and WITH CUTS (right)
\begin{figure}[t]
%
\center
%
\includegraphics[width=0.47\textwidth]{figures/spectrum_ee_truth.png} \includegraphics[width=0.47\textwidth]{figures/spectrum_ee_truth_cuts.png}
%
\caption{\label{fig:spectrum_ee} Characteristic spectra for the total energy of observed  $e^+e^-$ pairs seen at \muboone\ produced in the $\ster \to \nu e^+e^-$ decay mode. The three distributions correspond to parent sterile masses given in the legend. We see a preference for low energy events, with most events with energies below $0.4$ GeV. The modal peak of the distribution moves to higher energies as the mass of the sterile neutrino increases. On the left, the spectra have no analysis cuts or detector reconstruction effects applied, while on the right these are included, decreasing the discriminatory power of the lowest energy events.}
%
\end{figure}

\begin{figure}[t]
%
\center
%
\Large

\resizebox{0.8\columnwidth}{!}{\input{figures/spectrum_angles_ee.tex}}
%
\caption{\label{fig:spectrum_ee_angular} The angular spectrum of $e^+e^-$ events (defined as the direction of the sum of their individual momenta) in \muboone\ for a sterile neutrino mass of $m_N = 100$ MeV. The red histogram shows the true expected distribution, which we see is forward pointing. In blue we see the spectrum of events if we do not take into account the preferential decay rate for lower energy particles. We see that this probabilistic factor leads to a 
significantly less forward event spectrum.}
%
\end{figure}
\marcom{MARK}{Are the colors/ discription in this plot backwards? as in the Blue spectrum is more forward correct?}

%\begin{figure}[t]
%%
%\center
%%
%\includegraphics[width=0.47\textwidth]{figures/spectrum_ee_truth.png} \includegraphics[width=0.47\textwidth]{figures/spectrum_ee_smeared.png}
%%
%\caption{\label{fig:spectrum_ee} Characteristic spectra for the total energy of observed  $e^+e^-$ pairs seen at \muboone\ produced in the $\ster \to \nu e^+e^-$ decay mode. The three distributions correspond to parent sterile masses given in the legend. We see a preference for low energy events, with most events with energies below $0.4$ GeV. On the left, we see the events without any detector resolution effects, and the distribution has a peak at low energies which is seen to move to higher energies as the mass of the sterile neutrino increases. On the right, a Gaussian smearing has been applied to the total energy. Here we see that the distributions are more uniform but the ratio of events above $0.2$ GeV could serve as a mass sensitive parameter.}
%%
%\end{figure}
% 
%\begin{figure}[t]
%%
%\center
%%
%\includegraphics[width=0.47\textwidth]{figures/spectrum_ee_truth.png} \includegraphics[width=0.47\textwidth]{figures/spectrum_ee_situ.png}
%%
%\caption{\label{fig:spectrum_ee} Characteristic spectra for the total energy of observed  $e^+e^-$ pairs seen at \muboone\ produced in the $\ster \to \nu e^+e^-$ decay mode by decay in flight (left) and by \emph{in situ} production and decay (right). The three distributions correspond to parent sterile masses given in the legend. We see a preference for low energy decay in flight events, with most events with energies below $0.4$ GeV, and the modal peak moving to higher energies as the parent mass increases. Although the same trend is seen for the \emph{in situ} production, low-energy suppression arising from the initial neutral current cross section produces a significantly broader distribution.}
%%
%\end{figure}
%
%

The differerential distributions from heavy sterile decay tend to produce
distinctive low-energy distributions of events with an appreciably forward
direction. 
%
The tendency towards low energies is predominately due to the higher decay
rates of low-energy particles, which as can be seen in the previous section,
leads to factors of $1/E_\nu$ in the event rate formula \refeq{eq:prob_dec1}.
%
The forward trajectory in inherited from the kinematics of a boosted object
decaying in flight. However, this effect is mitigated by the preference for
lower energy decays, meaning that lower energy steriles are more likely to
decay, which are the least boosted objects.

We show an example of a distribution for electron-positron production in the
left panel of \reffig{fig:spectrum_ee}. For the lowest masses that we consider,
almost all events have energies below $0.5$ GeV, in this case illustrated by
the red histogram. The distribution tends towards larger energies as the mass
of the sterile increases, but for sterile masses less than the kaon mass, never
produces significant numbers of events about $1$ GeV. As we can see in
\reffig{fig:spectrum_ee}, the number of events in the lowest energy bin is
strongly indicative of the mass of the parent particle, and therefore the
lowest energy events will play a strong role in model discrimination. However,
as can be seen in the right panel of \reffig{fig:spectrum_ee}, in the cut-based
analysis which we outline in \refapp{app:bg} the lowest energy event
distribution is significantly reduced due to poor efficiencies at low-energy in
our cuts. Through optimisiation of this part of the analysis, we can expect
improvements in the sensitivity to these models; however, this is beyond the
scope of the present work. 



%
%Although both models produce visible particles through sterile neutrino decay
%in flight, the additional neutral current cross-section required by the
%Gninenko-type sterile suppresses the lowest energy events. As can be seen in
%\reffig{fig:XXX}, we see a marked difference between these signals at SBND. If
%we factor in the cut-based analysis that was performed in the preceeding
%section, we see that the model in this paper does lose a significant number of
%events from the lowest bins. To separate these events cleanly, the analysis
%should be optimised to keep as many of these lowest energy events as possible. 
%
%

\subsection{\label{sec:timing}Role of event timing}




\begin{figure}[t]
\center
\begin{subfigure}[c]{0.68\textwidth}
\includegraphics[width=\textwidth]{figures/timing.pdf} 
\end{subfigure}%
~
\begin{subfigure}[c]{0.32\textwidth}
\includegraphics[width=\textwidth]{figures/line_plots_long.pdf}
\end{subfigure}
\caption{\label{fig:timing} \emph{Left:} The timing delay of sterile neutrino
decays in nano-seconds for both a 25 MeV (top) and 350 MeV (bottom) sterile
neutrino at the SBND and and \icarus\ detectors (110 and 600m
respectively). The 4 ns beam bucket window is shown highlighted in red from 0
to 4 ns, followed by an additional 17 ns gap. The timings are calculated as a
difference to the time of flight of a active neutrino, assuming the decay
occurred in a uniform sample across the 50m BNB decay pipe. A timing resolution
of 1 ns is assumed to smear the observed events. \emph{Right:} The percentage of sterile decay events that
fall into the inter-bucket region as a function of sterile mass for SBND and
ICARUS, assuming a flux derived from $U_{e4}$ ($U_{\mu 4}$) mixing in solid
(dashed) lines. Both SBND and ICARUS see a sizable fraction of total events
outside the beam bucket windows when the sterile mass exceeds $\approx10$ MeV.
}

\end{figure}



Although the drift time of electrons in LAr can be as large as $\mathcal{O}$(100)$\mu$s, the ionization excitation of Argon from the passing of a charged particle also produces 128nm scintillation light of which there is a nano-second scale contribution from the excited state $Ar^*_2$ decaying \cite{Acciarri:2015hha}. LAr is transparent to this light so if the light detection system (LDS) employed by the SBN detectors has a similar nanosecond resolution, this can allow for percise timing to be attached to each TPC triggered event. Light neutrinos propagate and reach the furthest detector
of the SBN complex after approximately 2 $\mu$s. In the conventional physics programme
of the SBN, the timing of these events play an important role in the analysis
of backgrounds, tight timing windows are placed around the 19.2$\mu$s beam
spill to limit constant rate backgrounds such as cosmogenic events. The LDS of both SBND
and ICARUS, however, are expected to be able to achieve significantly better
timing resolution than this, $\approx \mathcal{O}$(1-2 ns) depending on the exact technology used, which potentially allows for the use of both bucket
and spill structure in the background analysis. The BNB consists of 81
Radio-Frequency buckets of approximately 2ns length, separated by 19 ns, to
form the 19.2$\mu$s spill with a frequency of 3Hz \cite{Antonello:2015lea}.
If this nano-second resolution is indeed achieved, it allows for events in individual buckets to be
identified. Such a nano-second resolution was achieved by the PMT's utilised in MiniBooNE \cite{Antonello:2015lea}, with potential for improvement in the next generation SBN detectors. \muboone\ is omitted from such a timing related analysis as the achievable timing
resolution is lower, $\approx 10$ns.

As particles with finite rest mass, these heavy neutral leptons will propagate
at subluminal speeds and for larger masses can produce observable timing
delays. This effect begins to become relavent when the sterile neutrinos are of
MeV-scale masses and above. As the flux of decaying sterile is inversely
proportional to its momentum after convoluting with their decay probability, many of these low energy steriles are
traveling at sufficiently slower speeds than their light counterparts to be distinguishable. Shown
in \reffig{fig:timing_line} below is the fraction of events that are expected
to fall outside the  bucket window in both SBND and ICARUS, with detailed
timing spectrums of a 25 and 350 MeV sterile shown in \ref{fig:timing}. For the purposes of this study we define the beam-correlated window to be a 4ns period, 1ns each side of the 2ns beam bucket.

As can be seen a significant proportion of sterile events are distributed throughout the inter-bucket area. Events which fall into the beam-bucket timing window have to be analysed on top of all known beam-related backgrounds, where as the increasingly larger fraction which populate the inter-bucket window have extremely reduced beam-correlated backgrounds.  

The implication of the timing effects is that our discussion of backgrounds is
twofold: for low mass steriles, or any mass at \muboone\ due to timing
resolution, we must consider all beam-related backgrounds as potential
backgrounds. For larger masses at SBND and ICARUS, we instead reduce the number of signal events by the appropiate fraction that lies in the bucket window and assume that the beam-related backgrounds are removed. 

\section{\label{sec:sensitivities}Sensitivities}

\subsection{Exclusion contours}
%
The results of our analysis are shown in \reffig{fig:band_sbn} below for the
combined exposure at all three detectors of the SBN facility. The lower colored lines show the
results of a backgroundless search, corresponding to the theoretical best
achievable sensitivity at the facility, assuming perfect signal efficiency. The
results of a simple cut based background analysis is also included as the upper
colored line. This background analysis is not optimsed, however, and serves
primarily to show the potential impact of backgrounds, see \refapp{app:bg} for details of the backgrounds included in this analysis. We
estimate the true achievable sensitivity in each channel would lie somewhere
between these two curves (the solid shaded region). The analysis only considers
the total number of events in each channel, with the 90\% C.L defined as $2.44$
events, following the procedure of \refref{Feldman:1997qc} designed for
backgroundless searches for rare events. For the background-included search the
representative quantity $S/\sqrt{B}$ is contoured for a direct comparason.  \\

We conclude that SBN sees more events that PS-191 and even when taking into account the increased backgrouds, if the timing and energy resolution is sufficiencly good, SBN could potentially improve the bounds for large sections of parameter space in all
channels studied. The previously unbounded $\ster \rightarrow \nu_\alpha \pi^0$ can potentially be bounded at a level very similar to the previous $e^+ e^-$ channel bounds as published by PS-191.  

\begin{figure}[t]
\center
\includegraphics[width=1.0\textwidth]{figures/band_sbn.pdf}

\caption{\label{fig:band_sbn}The sensitivity contours for the combined SBN
facility. Shown also in black solid lines is the prior best bounds from PS-191,
scaled to show bounds on the minimal extension as discussed here, as well as bounds from lepton peak searches in pion and kaon decay \cite{PhysRevD.46.R885,PhysRevLett.68.3000}(dashed black lines) . Note that the peak searches are only valid when bounding pure mixing combinations, e.g $|U_{\mu 4}|^2$ and not $|U_{\mu 4}||U_{\tau 4}|$ . In all panels, the mixing matrix elements not shown on the $y$-axis are zero. Above we
have the previously bounded channels, $e^+e^-$ and $l^- \pi^+$ for nonzero
$U_{\mu 4}$ and $U_{e4}$ respectively. Below we have the same for the $\gamma$ and $\pi^0$ channels. These channels have little or no direct bounds, with ISTRA+ bounding the radiative decay\cite{Duk:2011yv} and reinterpreted $\ster \rightarrow \nu \gamma \gamma$ bounds at NOMAD on $\ster \rightarrow \nu \pi^0$ \cite{Gninenko:1998nn} }

\end{figure}

\subsection{\label{sec:timing_physics}Timing for model independent mass measurement}

In addition to being able to reduce beam-related backgrounds by studying the
inter-bucket windows, the timing of the observed events can also be used to
discriminate between different models, as well as give an independant
measurement of the sterile mass that avoids the need for any normalisation. This is a very important tool, being valid in both the minimal model as described above but also in extensions in which the total rate has additional parameters allowing decoupling of the mass from the new degrees of freedom. 
Elementary special relativity tells us that for an arrival time delay (behind a luminal particle) over a
distance $L$ denoted by $\Delta T$, the mass of a sterile neutrino with an
energy $E$ can be reconstructed as 
%
\[ m_{\ster} = E\sqrt{1-\frac{1}{\left(1+\frac{c\Delta T}{L}\right)^2}}. \]

To determine the prospects for measuring $m_{\ster}$ using this information we
have generated Monte Carlo event data tagging each event by an arrival time.
We account for a systematic uncertainty associated with the time measurement as
well as the energy reconstruction. We smear energy to represent detector
effects as described above, and aditional smear the time of each event with a
finite time resolution of $\sigma_T  \approx 1$ for SBND and ICARUS. As the
absolute timing of the event is not known, only the relative timing since the
last bucket, $\Delta T$, one can obtain up to 81 degenerate solutions for the
sterile mass from any given event depending on the placement in he spill
structure. By only studying the initial few buckets one can elleviate the
problem, however, it also reduces the signal statistics by $\mathcal{O}$(0.01)
and so all leading and trailing edge effect information is ignored in this analysis. \\

To estimate the sensitivity of the decaying sterile mass we have run a binned Maximum Liklihood analysis of the reconstructed sterile energy and $\Delta T$, assuming events are poission distributed. As we do not wish to consider normalisation we assume that the experiment, here ICARUS, observes 100 events consistant with $\ster\rightarrow e^- \pi^+$ decay in flight.We define our test statistic as \[
	t_m = -2 \ln \left(\mathcal{L}\right) =  2 \sum_{i=1}^N \left[ \mu_i(m)-n_i +n_i \ln(\frac{n_i}{\mu_i(m)})  \right]
\]
where $\mu_i(m)$ is the expected number of events in bin $i$ if the sterile is of mass m, then the reconstructed mass defined as the mass which minimises $t_m$. As many bins contain low if any events, as well as the cyclic nature of $\Delta T$ we have elected to perform a Monte Carlo estimation of the distribution of $t_m$ simulating the entire experiment many thousand times in order to account for any non-gaussianity. The results of this study are shown in Fig (\ref{fig:tof_scatter}) below in the form of 1 and 2 $\sigma$ ranges of reconstructed sterile mass as a function of true simulated mass. Resolution of approx 40 MeV at 1$\sigma$ is achievable for the entire range of sterile mass allowed. The  $\ster\rightarrow \mu^- \pi^+$ channel is approximaly 10\% better due to the better energy resolution of muons in LAr. The remaining channels $\ster \rightarrow \nu_\alpha \pi^0$, $\ster\rightarrow \nu_\alpha e^+ e^-$ and $\ster \rightarrow \nu_\alpha \gamma$ have essentially no resolution via timing alone due to the inability to reconstruct the entire sterile energy.


\begin{figure}[t]
\center
\begin{subfigure}[t]{0.485\textwidth}
\includegraphics[width=\textwidth]{figures/sterilecomparason.pdf}
\end{subfigure}%
~
\begin{subfigure}[t]{0.515\textwidth}
\includegraphics[width=\textwidth]{figures/icarus_mass_sensitivity.pdf}
\end{subfigure}

\caption{\label{fig:tof_scatter}Left: The spectral differences total visible energy, $E_N^\text{RECO} \equiv E_{e^-}+E_{\pi^+}$, for sterile masses of 150, 250 and 350 MeV. Insert shows the stark differences between events falling within the beam bucket window and without.  Right: Sensitivity to sterile mass from reconstructed sterile energy only (Dashed and dotted lines) as well as when one includes additional time-of-flight information (purple shaded regions). Resolution of approx 45 MeV at 2$\sigma$ confindence is achievable for the entire range of sterile mass allowed if 100 events are observed. }

\end{figure}



\section{\label{sec:conclusions}Conclusions}

In this paper, we have studied its capabilities of the Fermilab Short-Baseline
neutrino facility to constrain decaying sterile neutrinos with masses around
the MeV scale. To make a fair assessment of the potential to constrain these
models, we have performed a simple cut-based analysis of the dominant
backgrounds and signals and shown that in conjunction with accurate timing
measuremnts high levels of background suppression can be expected resulting in
close to backgroundless searches in all channels studied. Using these
background estimates, we have performed a sensitivity analysis to the
parameters of the minimal extension of the SM with a novel sterile neutrino.
We have also motivated searches for non-minimal models, which in
particular could lead to observable decays over a wide range of parameter space
which is conventionally excluded by theoretical assumptions on the decay rates
themselves. We argue that these decay rates are actually \emph{unconstrained}
in published work, and show that the SBN could place the first direct bounds on
these processes. 

\acknowledgments

We would like to thank Andrezj Szelc for his input to various elements of this
work, and also to Jonathan Asaadi for helpful discussions at the start of this
project.

This work has been supported by the European Research Council under ERC Grant
“NuMass” (FP7-IDEAS-ERC ENC-CG 617143) and by the European Union FP7
ITN-INVISIBLES (Marie Curie Actions, PITN-GA-2011-289442).

\appendix

\section{\label{app:decayrates}Decay rates in the minimal model}

The dominant visible decay for sterile neutrinos with masses below the pion
mass is into an electron positron pair. The total rate can be express as
%
\begin{align*} \Gamma\left(\ster\to \nu_\alpha e^+e^-\right) =
\frac{G_\text{F}^2m_\ster^5}{96\pi^3}\left|U_{\alpha 4}\right|^2&\left[\left( g_Lg_R + \delta_{\alpha e}g_R\right)I_1\left(0,\frac{m_e}{m_\ster}, \frac{
m_e}{m_\ster}\right)\right.\\ 
&\left.\qquad + \left(g_L^2 + g_R^2 + \delta_{\alpha e}(1+2g_L)\right)I_2\left(0,\frac{m_e}{m_\ster},\frac{m_e}{m_\ster}\right)\right],  \end{align*}
%
where $g_L = -1/2 + \sin^2\theta_\text{W}$, $g_R = \sin^2\theta_\text{W}$ and
% 
\begin{align*} I_1(x,y,z) & =12 \int_{(x+y)^2}^{(1-z)^2}
\frac{ds}{s}(s-x^2-y^2)(1+z^2-s)\sqrt{\lambda(s,x^2,y^2)}\sqrt{\lambda(1,s,z^2)},\\
I_2(x,y,z)& =24yz\int_{(y+z)^2}^{(1-x)^2}\frac{ds}{s}\left(1+x^2-s\right)\sqrt{\lambda\left(s,y^2,z^2\right)}\sqrt{\lambda\left(s,y^2,z^2\right)},\\
\lambda(a,b,c) &= a^2+b^2+c^2 - 2ab-2bc-2ca.  \end{align*}
%

The decays into a charged lepton and a pion are given by 
%
\[ \Gamma\left(\ster\to l^\pm\pi^\mp\right) =
\left|U_{l4}\right|^2\frac{G_\text{F}^2f_\pi^2 |V_{ud}|^2
m_\ster^3}{16\pi}I\left(\frac{m_l^2}{m_\ster^2} ,
\frac{m_\pi^2}{m_\ster^2}\right) , \] 
%
with \[ I(x,y) = \left[ \left( 1+x+y\right) \left(1+x\right) -4 x\right]
\lambda^\frac{1}{2}\left(1,x,y\right).  \]
%
For $N\to e^\pm\pi^\mp$ the kinematic function $I(x,y)$ produces only weak suppression ($I(x,y)\geq 0.5$) for sterile masses above $m_\ster\gtrsim 150$ MeV, whilst for $N\to
\mu^\pm\pi^\mp$ the equivalent threshold is $m_\ster\gtrsim 270$ MeV.

\section{Potential Backgrounds\label{app:bg}}

In order to estimate the effect of expected backgrounds on the sensitivity we
have performed a brief Monte-Carlo analysis using the neutrino event generator GENIE
\cite{Andreopoulos:2009rq}. This provides us with generator level information about the
kinematics of the beam-driven backgrounds, with rates normalised off expected
NC and CC inclusive values as published in the SBN proposal. Energy and angular
smearing is then implemented to allow for approximate estimates of the effects
of detector performance to the level necessary for this analysis, without the
need for a full GEANT detector simulation. Energies are smeared according to a
Gaussian distribution around their true MC energies, with a relative variance
$\sigma_E/E = \xi/ \sqrt(E) $, where $\xi$ is a detector dependant resolution.
For this study we take the energy resolution for EM showers, muons and protons
to be 15\%, 6\% and a conservative 15\% respectively, alongside the angular
resolution of LAr to be $1.73^{\circ}$ \cite{Antonello:2015lea}. 

Of utmost importance in all studied channels to distinguishing signal and
background is the identification of a scattering vertex. Any hadronic activity
localized at the beginning of the lepton track is a smoking gun signal for beam
related scattering from deep-inelastic or quasi-elastic scattering. Therefore
any event containing one or more reconstructed protons or additional hadrons is
rejected outright. For counting this proton multiplicity we assume an detection
threshold of 21 MeV on proton kinetic energy in liquid Argon \cite{Acciarri:2014gev}, after smearing.
Background events that contain less than this and events that do not contain
any protons, such as events originating from coherent pion production, remain a
viable background and further rejection must come from the kinematics of the
final state particles only.

All two body visible decays discussed below have a powerful discriminator in
the reconstructed invariant mass of the charged particle pair, e.g  $M_{l^\pm
\pi^\mp}^2=m_l^2+m_{p^\pm}^2+ 2(E_l E_\pi - |P_l||P_\pi|\cos\theta_\text{sep})$
for $\ster\rightarrow \pi^+ l^-$, which sum to that of the the parent sterile
(within detector resolution), where as the background is expected form a broad
spectra spanning the energies of incoming neutrino. Additionaly, two body
decays allow for reconstruction of the parent sterile angle with respect to the
beamline which is assumed to be close to on-axis, unlike the more isotropic
backgrounds. We found $\approx 95$\% of reconstructed sterile from two bodys
decays were below $4^\circ$ relative to the beamline and we cut all events
greater than this. 

All three cuts defined above are applied to all events, both those that fall inside and outside the bucket timing window. Outside the timing window where one already expects extremely reduced beam related backgrounds these cuts reduce any the remaining to effectively zero. Inside the bucket timing window (and for all events at $\mu$BooNE) we look at further kinematic cuts to reduce potential backgrounds as elaborated below.

\begin{figure}[h]
\center
\includegraphics[width=0.6\textwidth,clip,trim=0 0 0 0]{figures/mu_pi_cutflow.pdf}

\caption{\label{fig:mu_pi_cutflow} Reconstructed sterile energy spectra for
CC$\nu_\mu$ backgrounds in comparison to a 350 MeV decaying sterile at
\muboone, normalised to 10 signal events. Total expected background of 98,013
events is reduced to $\approx$ 27 by successive kinematic cuts utilising
expected differences between decay and scattering behaviour. }

\end{figure}

\subsection{$\ster \rightarrow \mu^\pm \pi^\mp$ and $\ster \rightarrow e^\pm \pi^\mp$   }

The dominant backgrounds to the sterile decays we are interested in will be
genuine $\pi$-lepton production associated with the neutrino beam. So-called
CC1$\pi^+$ events are defined as the associated production of a charged pion
from the standard CC process which produces a lepton. These events can be
produced incoherently, often with large hadronic activity, or from coherent
scattering, where the neutrino scatters from the whole nucleus. These coherent
interactions tend to produce more forward decay products and will be another
significant source of backgrounds. Coherent cross-sections for these processes
have been studied in MiniBooNE \cite{Wascko:2006tx}, \minerva
\cite{Eberly:2014mra} and lately T2K and cross-sections appear to agree with
Monte Carlo calculations based on the Rein-Sehgal model \cite{Rein:2006di,
Rein:1982pf}.  Such a low $Q^2$ process tends to favour daughter pions and
muons that are forward going, kinematically very similar to decays in flight,
as well as no observable nuclear activity and as such are a potent
background.\\ 

The expected numbers of $e \pi$ events in the SBN detectors is significantly
smaller than that of the $\mu \pi$ channel, as the fraction of intrinsic
$\nu_e$ in the BNB beam is of $\mathcal{O}(1\%)$ level in comparison to
$\nu_\mu$. However, there will also be additional backgrounds to the $e \pi$
channel originating from the dominant $\nu_\mu$ beam. CC $\nu_\mu$ events which
contain an additional photon $(\mu+\gamma)$, or 1 reconstructed photon from a
$\pi^0$ decay, have the potential to be be mis-identified as an $(\pi e)$
event, provided the muon has a sufficiently short track length, $<$ 0.5 m, in
order to mimic a $\pi^-$. Additionally the photon must be mis-identified as an
electron, with an efficiency of 94\%, and furthermore, must convert to an
$e^+e^-$ pair close enough to the interaction vertex as so there is no visible
gap. Thus we accept 6\% of all CC $\mu+\gamma$ events whose photon converts
within 3cm, and whose muon track length is less than 0.5m. Combined with the
rejection of events with any vertex activity, this reduces the number of
background events to below that expected from intrinsic CC $\nu_e$ pion
production. As energy resolution for EM showers is lower than muons, the
invariant mass cut is less powerful requiring all events have an invariant mass
below 500 MeV. 

Additional kinematic cuts are selected to further decrease backgrounds. A cut
on lepton-pion opening angle, $\theta_{l \pi} < 40^\circ$ as well as indicidual
emission angles, $\theta_{l,\pi} < 80^\circ$, reduces the potential background
from 1,153,090 in SBND to approximately 323 events for $\mu^- \pi^+$.
\reffig{fig:mu_pi_cutflow} shows the background rejection capability of LAr as
each additional cut is implemented in \muboone, in comparason to a 350 MeV
sterile decay. The $e^- \pi^+$ channel is one of the cleanest channel under
consideration, with 9223 events in SBND reducing to 22 expected events post
cuts, and with \muboone\ and ICARUS expecting $\mathcal{O}(1)$ events from
beam driven backgrounds.


\subsection{$\ster \rightarrow \nu_\alpha e^+ e^-$ and $\ster \rightarrow \gamma \nu_\alpha$ }

A sufficiently boosted, and thus overlapping, $e^+e^-$ pair is topologically
indistinguishable from a converted photon in a LAr detector. Additional,
non-topological measures such as the rate of energy loss, $dE/dx$, is also
identical to a pair-converted photon. Thus we split this channel into two sub
categories, when the $e^+e^-$ is overlapping and photon-like, defined to be all
events whose angular separation is $\leq 3^\circ$\cite{Spitz:2011wba} and all
remaining separable two track events. The opening angle between the $e^+e^-$ in
a photon pair production scales roughly as $\approx m_e/E_\gamma$, with
$3^\circ$ corresponding to 100 MeV and used as a lower bound on energy. These
backgrounds are also applicable to the $\ster \rightarrow \nu_\gamma
\nu_\alpha$ channel.\\ 

Any standard process producing a lone stray photon is thus a possible source of
backgrounds for the photon-like $e^+e^-$ channel. The predominant source of
this in all three SBN detectors is the decay of a neutral pion in which a
single photon is not resolved or escapes the fiducial volume. This background,
however, is relatively isotropic in distribution and of lower energy that the
signal, in stark contrast to the very forward signal. Thus we apply a cut on
visible photon energy, $E_\gamma \geq 300 $ MeV and angle of the observed
photon to the beamline, $\theta_\gamma \leq 5^\circ$. In conjunction with the
requirement that no vertex activity is recorded, this reduced the number of
expected events from 42,580 to 176 events in SBND, while retaining a signal
efficiency of 93\%.

For the opposite scenario both daughter electrons have a well defined, and
large, separation and thus can cleanly be identified as two distinct single
electron showers. There are no significant processes that produce high energy,
distinguishable $e^+e^-$ pairs in a standrd neutrino beam.  Instead the
majority of the backgrounds are due to misidentifying photons, which are
abundant, as electrons. This can either be a single photon alongside a CC $e^-$
event or the more common NC $\pi^0$ production in which both of the underlying
photons are mis-identified as electrons. As both of these backgrounds require
the photons to mimic electrons originating from a single decay vertex, we also
require that all photons pair convert within 3cm of the start of the electron
track in the case of CC $\nu_e$ backgrounds, or both photons convert within 3cm
of each other in the case of NC $\pi^0$ backgrounds. Further more, each event
passing the above cuts is rejected with a 94\% efficiency using measurements of
the rate of energy loss, $dE\/dx$. All remaining events are kept as
backgrounds. To further reject backgrounds we apply a cut on angle of
separation between the distinct $e^-e^-$ tracks of $\theta_\text{sep}\leq 40
^\circ$ and total energy, $E_{e^+}+E_{e^-} \geq 100$ MeV. This reduces the
number of expected backgrounds from 173 events to 5 events in SBND. 

\subsection{$\ster\rightarrow \pi^0 \nu_\alpha$}

Although a sub-dominant decay mode when steriles mix with electrons alone, when
one considers non-zero $\vert U_{\mu4}\vert^2$ the branching ratio of $\ster
\rightarrow \nu_\mu \pi^0$ becomes dominant for a mass window $\approx 140
\rightarrow 240$ MeV. There is no known experimental publication of $\ster
\rightarrow \nu_\alpha \pi^0$ in the literature, mainly due to the large
expected backgrounds. Single neutral pions are produced in great numbers at the
three SBN facilities, so the lack of any nuclear recoil is crucial in
eliminating the incoherent neutral pion production background. As the $\pi^0$
itself is not visible, we focus on the reconstructed pion energy and angle,
inferred from measuring the resultant photons energy and angle, smeared as
usual to model detector effects. Only events in which both photons convert
inside the fiducial volume are accepted. SBND expects 127,211 $\pi^0$ events,
of which $\approx 602$ survive all cuts with a signal efficiency of 32\% for a
sample 350 MeV sterile. \\ 

To summerise, we collect the total numbers of beam-induced background events
expected at the three SBN below in \reftab{tab:Rates}, both before and after we
apply all visible vertex and kinematic cuts as described above. One can see
that LAr provides an impressive level of background rejection, with \muboone
and ICARUS being close to backgroundless in many channels, and SBND expecting
only $\mathcal{O}(100)$ events in all channels.

\begin{table}[t]
\centering
\begin{tabular}{ l | l |  l | l |  }
	Signal Channel & Events @ SBND & \muboone\ & ICARUS \\
\hline\hline
$\mu^\pm \pi^\mp$ &  1,530,900  & 98,013 & 164,716\\
													  w/ Cuts &323 & 27 & 46 \\ \hline
$ e^\pm \pi^\mp$ &  9,228  & 784 & 1,317\\
													  w/ Cuts &22 & 2 & 3 \\ \hline
$ \nu_\alpha \pi^0$ &   127,217 & 10,813 & 18,172\\
													  w/ Cuts &603 & 51 & 86 \\ \hline
$ e^+e^- \text{ (Separate)} $ & 173 & 14 & 24\\
													  w/ Cuts &5 & 0.3 & 0.5\\ \hline
$ e^+ e^- \text{ or } \gamma \text{ (Photon-like)}$ &  42,580 & 3,620 & 6,082\\
													  w/ Cuts &176 & 46 & 110 \\ 
 \hline \hline

\end{tabular}

\caption{\label{tab:Rates} A summary of the main sources of backgrounds for
each channel studied, before and after spectral and particle ID cuts are
applied as discussed in \refapp{app:bg}. }

\end{table}


\subsection{Non-Beam related backgrounds}
Cosmogenic backgrounds are a potential source of background for any analysis at SBN. In the case of cosmic muons, \icarus\ expects to
see approximately $2.5 \times 10^{6}$ cosmic events in the 211 second beam
spill, and are reduced to approximately 5 events expected after utilizing the
spill structure, scintillation light patterns and cuts on $\frac{d E}{d x}$
\cite{Antonello:2015lea}.  Alongside this impressive cosmic rejection, our
signal events are focused heavily along the beamline, hence we do not expect
cosmics to be a major source of background to any channel. In situ beam-off
cosmic studies will also allow potential backgrounds to be extremely well
understood by the time of an analysis such as this, are so are not included in
this analysis. 

\section{PS-191 Bound Repoduction\label{sec:ps191}}

As a consistency check of our methodology we reproduce here the bounds on
$|U_{e4}|$ and $|U_{\mu 4}|$ for sterile masses below $m_\pi$ as published by
PS-191. The detector geometry is assumed to be $6\text{m} \times 3\text{m}
\times 12 \text{m}$ and was located 128m downstream of the Beryllium target
using 19.2 GeV protons from the PS proton beam.  Fluxes of all neutrinos
produced from pion sources at PS-191 were obtained from \cite{ps191THesis}. No
accurate kaon sources could be obtained and as such only low mass bounds are
reproduced here. It must be noted that PS-191 ignored all neutral current
contributions to $\ster \rightarrow \nu_\alpha e^+ e^-$ and assumed the sterile
neutrinos were Dirac particles; the effect of this is that the bounds published
are not directly comparable to the minimal model discussed above, and must be
scaled appropiately. The bounds reproduced are in good agreement with published
data.

\begin{figure}
			  \centering
			 
\includegraphics[width=0.5\textwidth]{figures/PS-191_test.pdf}

\caption{\label{fig:ps191test} Estimated bounds on $|U_{e4}|^2$ and $|U_{\mu
4}|^2$ for a Dirac heavy sterile neutrino decaying to $\nu_\alpha e^+ e^-$ at
PS-191. The dotted black lines are the 90\% C.L results as published by PS-191,
and the blue and red curves are the results of our simulation for $0.86 \times
10^{19}$ POT.}

\end{figure}

%\newpage
%
%\section{To do list}
%
%\begin{enumerate}
%
%\item Do a bit more checking that the `unobserved' channels are indeed unobserved. 
%
%\item Are the effective model decay channels unbounded? \newtext{PB}{I think } \refref{Aparici:2009fh} \newtext{PB}{goes some of the way to placate my worries: they show bounds for their operators, and in our region of interest there is nothing killer (I think).}
%
%\item Turn on $U_{\tau 4}$,$U_{\mu 4}$ and $U_{e4}$ simultaneously. Think of
%how we can push the analysis beyond the obvious by using flavour effects. Also
%look at ratios of channels and ratios of events to see if we can get any
%(broken) degeneracies that could actually be interesting \emph{i.e} SNO style
%complementary measurements.
%
%\item \newtext{MARK}{Working on} Make nice version of branching ratio plot.
%Need to make prettier. \sout{Also, we probably need to cut it off at
%$M_\ster\lesssim 500$ MeV, there are other decays allowed between 0.5 and 1 GeV
%(with $\eta$ and $\rho$)}.
%
%\item \newtext{MARK}{Working on} Think if there's a nice plot showing the
%sterile effect on our fluxes.
%
%\end{enumerate}
%

%%%%%%%%%%%%%%%%%%%%%%%%%%%%%%%%
%%%%%%%%%%%%%%%%%%%%%%%%%%%%%%%%
%%%%%%%%%%%%%%%%%%%%%%%%%%%%%%%%

\bibliographystyle{apsrev4-1}
\bibliography{lib}{}

\end{document}

